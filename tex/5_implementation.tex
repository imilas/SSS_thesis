% Allow relative paths in included subfiles that are compiled separately
% See https://tex.stackexchange.com/questions/153312/

\providecommand{\main}{..}
\newcommand{\decfirst}{\textit{Decision.1}}
\newcommand{\decsecond}{\textit{Decision.2}}
\documentclass[\main/thesis.tex]{subfiles}

\externaldocument{}


\begin{document}
\chapter{Making Noise, Hearing Drums}
\label{implementation}


\section{What Are We Doing Again?}
Our goal is to create a system for the generation of novel drum sounds. So far, we've discussed what tools to pick for virtual synthesis, and how to represent digital sounds in a smaller, more \enquote{learn-able} format. In this chapter, we will discuss the two main components of our system: The \textit{virtual synthesizer} and the \textit{virtual ear}. A quick outline of our generative pipeline is this: The virtual synthesizer will rapidly generate random programs and the corresponding sounds. As the virtual synthesizer creates sounds based on these programs, the virtual ear will listen to the sounds and determine if they should be categorized as drums and if so, which category of drum do they belong to. 

A visual representation of this pipeline is illustrated in  figure~\ref{fig:pipeline_outline}.In Section~\ref{vs} we discuss the implementation of a virtual synthesizer which can create a wide variety of noise. In section~\ref{sec:ear}, we discuss the implementation of the virtual ear, its task being the separation of percussive sounds from non-percussive sounds.


 \begin{figure}[t!]
    \begin{center}
    \textbf{Pipeline Design}
    \makebox[\textwidth]{
    \fbox{\includegraphics[width=1.1\linewidth]{images/chapter_4/pipeline.pdf}}}
    \end{center}
    \caption{A blueprint of our desired pipeline which allows each component to be implemented in a number of ways. The implementation of this workflow also allows for easy parallelization when needed.
    }
\label{fig:pipeline_outline}
\end{figure}


\section{Virtual Synthesizer}
\label{vs}
 To create sounds, we need digital synthesizers capable of rapidly receiving or creating programs and rendering the corresponding sound offline. We use classical DSP to build the synthesizer, which allows for quick, offline, and parallel generation of audio signals without the usage of GPUs. We made extensive use of Pippi\footnote{https://github.com/luvsound/pippi} and SciPy~\cite{jones2001scipy} libraries. The virtual synthesizer contains a set of one or more submodules. Each submodule is a self-contained noise making unit and creates signals by taking the steps depicted in figure~\ref{fig:submodule}. Submodules have identical sets of parameters, but widely different outputs can be achieved depending on the values assigned. The set of required parameters for each submodule is highlighted in Table~\ref{table:submodule_params}. The sonic output of the virtual synthesizer is the normalized addition of the output of its submodules. The synthesizer can have any number of submodules. We call the number of submodules in each virtual synthesizer the \textit{stack size}. We call the sets of parameter values that characterize a synthesizer's submodules a \textit{program} (analogous to presets and submodules for a VST).  

 \begin{figure}[htbp]
    \begin{center}
    % \textbf{Synthesizer SubModule }
    \makebox[\textwidth]{
    \fbox{\includegraphics[width=1\linewidth]{images/chapter_3/synthesizer_block.pdf}}}
    \end{center}
    \caption{High level representation of pre-rendering steps for each submodule. Each Synthesizer contains 1 or more submodules. Synthesizer programs set the number of these submodules and their parameters.
    }
\label{fig:submodule}
\end{figure}

 \begin{figure}[htbp]
    \begin{center}
    % \textbf{Synthesizer SubModule }
    \makebox[\textwidth]{
    \fbox{\includegraphics[width=1\linewidth]{images/chapter_3/synthesizer_all_blocks.pdf}}}
    \end{center}
    \caption{The output of the virtual synthesizer is the normalized addition of the output of its submodules. A synthesizer can have any number of submodules. 
    }
\label{fig:synth_modules}
\end{figure}
 Each submodule is a self-contained noise making unit. Submodules have identical sets of parameters, but widely different outputs can be achieved depending on the values assigned to these parameters. The set of parameters available to each submodule is highlighted in Table~\ref{table:submodule_params}. The sonic output of the virtual synthesizer is the normalized addition of the sonic output of its submodules. Our implementation of a synthesizer can have any number of submodules. The parameters that dictate the output signal of each submodule as well as the range of values each parameter can take are shown in table~\ref{table:submodule_params}. We call the number of submodules in each virtual synthesizer the \textit{stack size}. We call the sets of parameter values that characterize a synthesizer's submodules a \textit{program} (analogous to a preset for a VST).  

As we are interested in short, one-shot percussive sounds, each virtual synthesizer program will generate a 1 second piece of audio. This 1 second limit is over twice the length of the average one-shot drum sample in MixedDB. Each submodule can make an audio signal with the length of 0.1-1 second, and play it at any point within the 1 second rendering time (but the entire sound must fit within the second, that is, a 0.5 second sound cannot begin playing after 0.5 seconds within rendering time frame). If the synthesizer has a stack size of more than 1 the audio signals from each submodule are overlapped and the total amplitude is normalized.

\begin{table}[t!]
\centering
\resizebox{\columnwidth}{!}{\begin{tabular}{ |c|c|c| } 
\hline
Parameters & Value Range & notes and constraints\\
\hline \hline
Attack & 0-3 & A-D-S-R values relative\\
Decay & 0-3 & relative to A-S-R\\
Sustain & 0-3 & relative to A-D-R\\
Release & 0-3 & relative to A-D-S\\
OSC type & sine,square,saw & tone type\\
IsNoise & boolean & whether to \newline use OSC type to generate noise\\
Length & 0-1 second & - \\
StartTime & 0-1 second & Length+Start$<$1\\
Amplitude & 0.1-1 & 1 = max amplitude\\
Pitches(notes) & list of pitches &  range of C0(16.35hz) to B9 \\
HP filter Cutoff & 0-20000hz & -\\
LP filter Cutoff & 20000-HP & never lower than HP cutoff\\
Filter Order & 4,8,16 & butterworth filter order \\
\hline
\end{tabular}}
\caption{Synthesizer submodule Parameters. Despite the simplicity of the parameters and our efforts at constraining the ranges, the number of parameters that can be randomly chosen for each submodule is in the order of $10^{15}$ }
\label{table:submodule_params}
\end{table}
The ADSR parameters shape the entire amplitude of the signal. Each submodule creates its signal at full amplitude then shapes it according to its internal ADSR parameter. Prior to being applied to the signal, each of these parameters is assigned an integer value in the range of 0-3, and normalized relative to the others such that \[ A_{norm} + D_{norm} + S_{norm} + R_{norm} = 1 \] \\ 
Where each value $v_{norm}$ in the $\{A_{norm}, D_{norm},S_{norm},R_{norm}\} $ set is normalized such that:
\begin{align*}
\text{for each $v$ $\epsilon$ \{A,D,S,R\}} \\
v_{norm} = \dfrac{v}{A + D + S + R}
\end{align*}
 The OSC type will determine the wave-shape of the signal. We limited this parameter to three fundamental wave forms: sine waves, square waves and saw waves. We also allow the creation of noise signals, which can imitate timbral characteristics of higher pitched drum samples at a very low computation cost (relative to the addition of thousands of sine waves at various frequencies). If the IsNoise boolean is set to true, the OSC type parameter loses importance as the OSC type will simply be used for the generation of noise via random wave-table transformations. Before the filter and ADSR envelope take affect, the generated noise will have similar characteristics to white noise. 

Each submodule is a monophonic synthesizer. That is, each submodule can play one note (or frequency) at a time. However, quick changes in pitch can occur in drum sound. Our more successful attempts at creation of synthetic kick or bass drums are often characterized by quick, exponential decrease of a sine-wave's pitch from medium to low audible frequencies. To mimic such sounds, synthesizer submodules may slide between 4 different pitches in the 1 second time frame. Each pitch value is a midi note with a frequency and length value. Each submodule accepts a list of 5 consecutive possible pitch values. The submodule will play each note in the list consecutively after normalizing the length values. The pitch notes are played in a portmanteau fashion such that there is no audible gap. This normalization of length values is similar to that of the ADSR values. 


\subsection{Is the Virtual Synthesizer Deterministic?}
\label{chap3:synth_deterministic}
It is crucial that the same set of synthesizer parameters result in identical, or near identical sounds. Our synthesizer modules are capable of producing random waves, or \enquote{noise}. This type of sound will vary each time it is produced. Manual listening does not show sonic variation between multiple creation of the same set of parameters, but how does this affect our ear models?

We create 50 parameter sets for stack sizes of 1 and evaluate each parameter set 10 times using the CNN-DVN model. This model measures the probability of a sound belong to the drum/percussion category, and will be discussed in more detail in Section~\ref{TPE_models}. Put simply, we are creating the sonic output of the same parameter set multiple times and measuring its effect on one of our models. We repeat this experiment for stack sizes of 2 and 8 and show the results in figure~\ref{fig:synth_deterministic}.

\begin{figure}[htbp!]
\begin{center}
    \textbf{ Variation in Repeated Evaluations of Parameters }\par\medskip
    \makebox[\textwidth]{
    \subfloat[1 Stack]{ \includegraphics[width=17cm,height=5.67cm]{images/chapter_3/eval_stability_stacksize1.pdf}

    }}

    \makebox[\textwidth]{
    \subfloat[2 Stacks]{ \includegraphics[width=17cm,height=5.67cm]{images/chapter_3/eval_stability_stacksize2.pdf}

    }}
    
    \makebox[\textwidth]{
    \subfloat[8 Stacks]{ \includegraphics[width=17cm,height=5.67cm]{images/chapter_3/eval_stability_stacksize8.pdf}

    }}
\end{center}

\caption{Repeated evaluation of signals from identical paramter-sets shows little to no variation in scores. Multiple renditions of the same parameter-set vary if the parameter-set calls for generation of 1 or more random wave-shapes.}
\label{fig:synth_deterministic}
\end{figure}


\section{The Ear}
\label{sec:ear}
What we refer to as an \enquote{ear} is any method of scoring and classifying audio (e.g machine listening) \cite{malkin2006machine,rowe1992interactive}. The ideal ear for our task will be capable of receiving a piece of audio and giving it a score (or a list of scores) based on how well it satisfies certain criteria. In this work we are mainly focused on generation of novel percussive sounds, therefore, what we require from the ear is to give us probabilities of an audio sample belonging to various categories. Our synthesizer outputs are close to deterministic for all programs, therefore evaluations of the ear would allow us to associate scores of the sound with the parameter values (or program) that generated it.

We are not looking for the perfect imitation of organic drums using a synthesizer. We seek to imitate drums using a synthesizer and even \textbf{prefer} for its generations to retain novel, unusual characteristics. The task assigned to the virtual ear is the rare acceptance of drum-like sounds and the inevitable rejection of most \enquote{noise} outputs from the virtual synthesizer. By definition, we cannot precisely anticipate what novel drums will sound like. All we know is that the accepted sounds are synthesizer noises which the synthetic ear has determined to be similar to drum-sounds. 



% we denote $\mathcal{N}$ as the set of percussive sounds a synthesizer is capable of making. $\mathcal{T^{+}}$, our positive examples of what percussive instruments sound like is a set of sound extracted from material drums. $\mathcal{T^{-}}$, is a small subset of an infinite set which is meant to represent all noises which the synthesizer is capable of making. It likely includes a number of sounds which to our ears could be used as drums. During the hearing test, the virtual ear continually receives synthesizer noises ($\mathcal{H}$) for classification. The performance of the ear is measured by its precision and recall in finding the overlap between $\mathcal{H}$ and $\mathcal{N}$.

Figure~\ref{fig:ven_data} highlights a critical problems with this approach. The change in learning domains---particularly with the case of positive examples of percussive instruments---should not interfere with transformation of knowledge from the training of the ear to the hearing test.  \\
\begin{figure}[]
    \begin{center}
    \textbf{Data Overview}
    \makebox[\textwidth]{
    \includegraphics[width=1\linewidth]{images/chapter_4/venn_data.pdf}}
    \end{center}
    \caption{ An illustration of the discrepancy between the sounds we use to train our classifiers and the type of sounds the classifier is expected to classify. $\mathcal{N}$ is the set of percussive sounds a synthesizer is capable of making. The inclusion of sounds in this group may vary from person to person. Our positive samples, $\mathcal{T^{+}}$, is a small fraction of a wide variety of percussive sounds that are conceivable. For $\mathcal{T^{-}}$, we can generate any number of random samples. $\mathcal{H}$ is a series of sounds sent to the ear for classification.}
\label{fig:ven_data}
\end{figure}
 With our dataset of labeled drum sounds, discussed in Section~\ref{sec:data}, we implemented several well performing algorithms for categorizing unlabeled drum sounds, given that we know they are drum sounds. However, a major hurdle towards the implementation of a \textquotedblleft drum from non-drum\textquotedblright recognizer is that the set of sounds that are not percussive is infinite. Drum groups are an example of closed sets, since we believe that a sufficiently large sample pack can effectively describe common drum categories. However, effective representation of all possible non-drum sounds is not attainable via examples alone.

Traditional classification tasks often make the assumption the data points used for training the model and future unlabeled data will emerge from the same system of processes~\cite{geng2020recent,mundt2019open}. This assumptions requires that sufficient positive examples of all possible classes exist and are trained on. Works which involve the implementation of GANs have documented scenarios in which networks will assign high categorization probabilities to nonsensical, out of context data which should be rejected rather than categorized~\cite{geng2020recent,mundt2019open,hassen2020learning}. \\

 Having considered these caveats, once a sound is generated and passed onto the ear, we expect the virtual ear to make decisions in response to two important questions: 
\emph{
\begin{quote}
\text{Decision.1 Could the sound be used as a drum?}\label{Decision.1}
\\
\text{Decision.2 If it does sound like a drum, what type of drum should it be?}\label{Decision.2}
\end{quote}
    }
\decfirst~requires knowledge of what drums \textbf{do not} sound like, or knowledge of an infinitely large set, which cannot be fully represented via examples. An important consideration is that the source of sounds used for training the model (organic drum sounds) will be fundamentally different from the source of unlabeled sounds we wish to categorize (noise from a synthesizer). This issue is reflective of the open set recognition (OSR) problem~\cite{geng2020recent,mundt2019open}. We use Figure~\ref{fig:ven_data} to highlight a number of caveats with our training approach. If the sound is deemed percussive, the virtual ear makes \decsecond~by finding the best drum category for the sound. The number of categories available is dependent on the database of drums used for training. 
To maximize the transference of knowledge gained from training the classifiers to evaluation of programs, we need to extract concise feature sets that capture fundamental characteristics of the data points.

Our goal is to create a pipeline of sound generation where the synthesizer is used for the rapid generation of sounds and the virtual ear is used for the acceptance of inputs which satisfy some fundamental characteristics of percussive instruments. How we characterize this description is critical as it allows novel sounds to be accepted as part of the drum group despite their anomalies. In the previous chapter, we discussed 2 different approaches to feature extraction. This leads to two different implementation of a virtual ear: \emph{two phased ears} (TPEs) and \emph{mixed ear models} (MEMs). TPEs are a combination different models for each of \decfirst~and \decsecond. The features utilized by these models are manually defined. MEMs use a highly compressed, automatically encoded representation of sound to give simultaneous answers to both questions.


\subsection{Two Phased Ears}
\begin{figure}[t!]
    \begin{center}
    \textbf{TPE Design}
    \makebox[\textwidth]{
    \fbox{\includegraphics[width=1.1\linewidth]{images/chapter_4/TPE_ear.pdf}}}
    \end{center}
    \caption{TPE's receive a sound and make decisions sequentially.}
\label{fig:TPE_design}
\end{figure}


\label{TPE_models}
Using the features and transformation described in the previous chapter as input, we trained several neural network models with the pytorch library. The task at hand with regards to \decfirst~is to separate drums from not-drums (DrumVsNotDrum, or DVN). In \decsecond~the aim is to categorize the type of drums and percussion (DrumVsDrum, or DVD). Here we cover the architecture, training, and accuracy of the TPN models.

 For TPEs, we train multiple neural network architectures using different subsets of the FFT features to specialize in \decfirst~or \decsecond~and combine them to make decisions sequentially. To train TPEs for \decfirst, we use all drums in RadarDB and FreeDB and 6000 examples of virtual synthesizer noise. For \decsecond, we combine the two databases and merge toms into kicks and rims/shakers into \enquote{other}. We trained the TPE models with 80\% of this dataset. Using the remaining 20\% of sounds, we achieved 98\% accuracy in \decfirst~and 82\% accuracy in~\decsecond~with our best models.  The loss function and optimizer are Categorical Cross-Entropy and Adam respectively. Training continues until no reduction in loss and accuracy is observed in 10 epochs. These accuracy numbers are weak as we did not account for category sizes or cross validate.  


\subsubsection{DVN Models}
\textbf{FC-DVN}: Fully connected network trained on Envelope features, reaching 97\% accuracy on our test data for~\decfirst. 
\begin{center}
\vbox{
    \begin{lstlisting}[caption=Sequential Layers for CNNLSTM-dvn]
            Layer                       Details
            ========================================================
            Linear                      in_features=400 
                                        out_features=20 
                                        bias=True
            PReLU                       
            Linear                      in_features=20
                                        out_features=10 
                                        bias=True
            PReLU
            Linear                      in_features=10
                                        out_features=4
                                        bias=True
            PReLU
            Linear                      in_features=4
                                        out_features=2
                                        bias=True
            Softmax                     DVN probabilities
                        
        \end{lstlisting}}
\end{center}
\textbf{CNNLSTM-DVN}: A combination of CNN and LSTM models, where the CNN model extracts higher level features that are fed temporally to an LSTM cell. This model is trained on spectrum data and reaches 98\% accuracy on our test set. 
\begin{center}
\vbox{
    \begin{lstlisting}[caption=Sequential Layers for CNNLSTM-dvn]
            Layer                       Details
            ========================================================
            Conv2d                  kernel size = (7, 3)
                                    stride = (1, 1)
                                    padding = (3, 1)
            ReLU
            Dropout                 probability = 0.5
            LSTMCell                hidden size = 800
                                    input size = @($\mathcal{F}$)@
            Linear                  in_features = @($\mathcal{F}$)@
                                    out_features = 2
                                    bias=True
            Softmax                 DVN probabilities
\end{lstlisting}}
\end{center}

\subsubsection{DVD Models}
\textbf{E+F-DVD}: A fully connected model trained on a concatenation of envelope and frequency features. Reaching 80\% accuracy for 6-way drum categorization in Phase 2. 
\begin{center}
\vbox{
    \begin{lstlisting}[caption=Sequential Layers for E+F-DVD]
            Layer                       Details
            ========================================================
            Linear                      in_features = 10+50 
                                        out_features = 30
                                        bias = True
            PReLU                       
            Linear                      in_features = 30
                                        out_features = 10
                                        bias = True
            PReLU                       
            Linear                      in_features = 10
                                        out_features = 10
                                        bias = True
            PReLU     
            Linear                      in_features = 10
                                        out_features = @($\mathcal{C}$)@
                                        bias = True
            Softmax                     drum type probabilities
        \end{lstlisting}}
\end{center}

\textbf{CNN-DVD}: A CNN model trained on Spectrum features. Reaching 82\% accuracy in a 6-way drum categorization in Phase 2.   {\color{red}double check architecture}
\begin{center}
\vbox{
    \begin{lstlisting}[caption=Sequential Layers for CNNLSTM-dvn]
            Layer                       Details
            ========================================================
            Conv2d                  kernel size=(4, 5)
                                    stride=(1, 1)
                                    padding=(3, 1)
            ReLU
            Dropout                 probability=0.5
            Conv2d                  kernel size=(8, 3)
                                    stride=(1, 1)
                                    padding=(3, 1)
            Linear                  in_features=100
                                    out_features=@($\mathcal{C}$)@
                                    bias=True
            Linear                  in_features=20
                                    out_features=@($\mathcal{C}$)@
                                    bias=True
            Softmax                 drum type probabilities
        \end{lstlisting}}
\end{center}

\textbf{FC-DVD}: Fully connected 3 layer neural net with 78\% accuracy for 6-way drum categorization in Phase 2. 
\begin{center}
\vbox{
    \begin{lstlisting}[caption=FC-DVD]
            Layer                       Details
            ========================================================
            Linear                      in_features=400
                                        out_features=20 
                                        bias=True
            PReLU                       
            Linear                      in_features=20
                                        out_features=10 
                                        bias=True
            PReLU
            Linear                      in_features=10
                                        out_features=4
                                        bias=True
            PReLU
            Linear                      in_features=4
                                        out_features=@($\mathcal{C}$)@
                                        bias=True
            Softmax                     drum type probabilities
                        
        \end{lstlisting}}
\end{center}
\subsubsection{Creating An Ear}
These models can be combined and weighted in various ways and the confidence thresholds can be modified in order to implement \enquote{virtual ears} with different properties. A glaring issue in the current implementation is the treatment of softmax outputs as a reasonable measure of a model's confidence. As a result, some models may have unwarranted higher confidence in their scores, skewing attempts at finding a consensus.  

With our models showing high accuracy on testing data, we combine models in order to make decisions sequentially. For \decfirst~we only determine sounds as percussive if both FC-DVN and CNN-LSTM have determined it as such with over 90\% confidence. For the majority of our random generations that is not the case, but if a randomly generated sound has passed this phase, our three categorizers assign their categorizations to this sound. The fourth method of categorization, \enquote{averaged-cat}, is implemented by taking the sum of the softmax outputs of all three categorizers, using it to determine the category.  These categorizers have a moderate degree of agree-ability as seen in section ~\ref{surveys}, but often the decision is not unanimous.




\subsection{Mixed Ear Models}

\begin{figure}[t!]
    \begin{center}
    \textbf{MEM Design}
    \makebox[\textwidth]{
    \fbox{\includegraphics[width=1.1\linewidth]{images/chapter_4/MEM_ear.pdf}}}
    \end{center}
    \caption{MEMs use both FFT features and embedding features to make both decisions simultaneously. }
\label{fig:TPE_design}
\end{figure}
\label{chap3:mixed_ear_models}
In Section~\ref{section:embedded_feats}, we discussed embedded features and visualized their potential using t-SNE. As t-SNE results are not classifiers in of themselves, we create a model which uses encoded versions of each sound group to predict its type. This differs from two-phased ears as we are simultaneously categorizing a new sound's drum type or putting it in the \enquote{synthetic noise} category. We call this task drum vs drum vs not-drum, or \enquote{DvDvN}. We also tackle \decfirst~using embedded features, although DvN MEM models are not used in the final pipelines, as we can make both decisions simultaniously.

We mentioned previously that manual t-SNE inspections highlighted the disregard for envelope shapes as a major source of failure. We compare the performance of our models before and after the addition of envelope features (a vector of size 10) to the feature space. 

We reuse the model trained on  MixedDB as our embedding feature extractor. We use a combination of RadarDB, FreeDB and NoiseDB for training. We only focus on clap,hat,kick,snares and synthetic noise groups for measuring effectiveness to prevent class overlaps as much as possible. We also exclude samples longer than 1 second, to reduce potentially mislabeled data. Our final training database for mixed ear models is described in Table~\ref{db:memDB}.
\begin{table}[htbp]
\centering
\begin{tabular}{|l|l|l|l|l|l|}
\hline
 DB Name & kick & snare & clap & hat & Synthetic Noise\\\hline
 MixedEarDB & 1334 & 1035 & 401 & 1275 & 1000 \\ \hline
\end{tabular}
\caption{MixedEarDB: A database put together by combination of radarDB, FreeDB and NoiseDB}
\label{db:memDB}
\end{table}

\subsubsection{Model Selection}
Using our encoding and envelope features to represent audio, five classification models were trained for the task of categorizing the five different sound groups. For hyper-parameter optimization, the models were trained using 5-fold cross validation and 80/20 train-to-test ratio. The F-Score result of each cross-validation is the unweighted average F-Score of all groups. For inter-model comparisons, the procedure is the same except 10-fold cross validations are used. Our models were derived from scikit-learn's implementations of these classifiers~\cite{pedregosa2011scikit}. Before inter-model comparisons, we conducted a grid-search for each model on at least one of its possibly decisive hyper-parameters. The classifiers and other notable specifications are presented in Table~\ref{table:mem_model_selection}. We introduced class weights where possible to mitigate the effects of an imbalanced dataset~\cite{provost2000machine,chawla2004special}. When utilized, the weight for each class $c$  is calculated as:

\begin{subequations}
    \begin{align*}
    c_{weight} = 1-\dfrac{\text{number of samples in group $c$} }{\text{Total number of samples}}
    \end{align*}
\end{subequations}

\begin{table}[t]
    \centering \hspace*{-0.8cm}
    \begin{threeparttable}
    \begin{tabular}[width=0.95\paperwidth]{|l|l|l|}
    \hline
    Model name & Tuned Parameters\tnote{\dag}  & Used Weights? \tnote{\ddag} \\\hline
     Support Vector Classifier (SVC) &  Gamma:0.001, C:100, kernel:rbf & Yes\\
     LinearSVC & C:10 & Yes\\
     K Nearest Neighbors & Num. Neighbors:31 &  No \\
     Random Forest Classifier & Num Estimators:500 & Yes \\
     Extra Trees Classifier & Num Estimators:1100 & Yes\\
     \hline
    \end{tabular}
    \caption{Models implemented for comparison using envelope and embedded features. }
    \begin{tablenotes}
    \item[\dag] Tuned parameters values are based on grid-searching for best f-score. Parameters not mentioned have neither been tuned nor changed from scikit-learn's default values (as of version 0.23)
    \item[\ddag] Class weights are used unless not applicable to classifier.
    \end{tablenotes}
    \label{table:mem_model_selection}
    \end{threeparttable}
\end{table}

\begin{figure}[htbp!]
    \begin{center}
    \textbf{Cross Validation F-Scores For All Sound Groups}\par\medskip
    \makebox[\textwidth]{\includegraphics[width=0.8\paperwidth]{images/chapter_3/mme_comparisons_mme.pdf}}
    \caption{Boxplots visualizing the F-Score results for each cross-validation. The individual scores, means, medians, standard-deviation and outliers are depicted. The differences are noticeable, yet means lie within the \%88-92 range. Envelope features improve classification accuracy for all models. }
    \label{fig:f1_allg_box}
    \end{center}
\end{figure}

We're also interested in how these models perform on the binary drum vs not drum task.
After grouping all drums together, we repeat the model selection process above. We also repeat the hyper-parameter optimization step where no changes appeared necessary except for a reduction in the number of neighbors for the K-Neighbor model (from 30 to 5). As show in in figures~\ref{fig:f1_allg_box} and~\ref{fig:f1_dvn_box}, the addition of envelope features had a positive effect on performance for all mdoels, yet the RandomForest and ExtraTrees models clearly outperform the other classifiers in both tasks. We train the top two models on \%80 of our database and use the remaining \%20 to create the confusion matrices and the F-Scores shown in figures~\ref{fig:conf_f1_dvn}.
\begin{figure}[h!]   
    \begin{center}
        \textbf{Cross Validation F-Scores For Drum Vs Not-Drum}
    \makebox[\textwidth]{\includegraphics[width=0.8\paperwidth]{images/chapter_3/mme_comparisons_dvn.pdf}}
    \caption{F-Score results for each cross-validation. Models perform better as there are less categorization groups. Envelope features increase accuracy for all models. Random Forest and Extra Trees remain the top two models. }
    \label{fig:f1_dvn_box}
    \end{center}
\end{figure}


\begin{figure}[htbp!]
\begin{center}
    \textbf{ Classification Report for DvDvN and DvN  }\par\medskip
    \makebox[\textwidth]{
    \subfloat[Precision, recall, F1-Score, and number of supporting examples]{ \includegraphics[width=9cm,height=9cm]{images/chapter_3/f1_mme.pdf}
    \includegraphics[width=9cm,height=9cm]{images/chapter_3/f1_dvn.pdf}
    }
    
    }
\label{fig:conf_f1_dvd}
\end{center}

\begin{center}
    \makebox[\textwidth]{
    \subfloat[Confusion matrices]{ \includegraphics[width=9cm,height=9cm]{images/chapter_3/conf_mme.pdf}
    \includegraphics[width=9cm,height=9cm]{images/chapter_3/conf_dvn.pdf}
    }}
\end{center}


\caption{F-Scores and confusion matrix of ExtraTrees model for both DvDvN and drum vs not-drum categorization.}
\label{fig:conf_f1_dvn}
\end{figure}

Based on these reports, having multiple options for drum categorization does not noticeably influence our models accuracy in categorizing synthetic noise as synthetic noise. However, the DvDvN models's slightly smaller false negative rate for synthetic noise (11 vs 13 false negatives) is countered by a slightly higher rate of categorizing drums as synthetic noise (12 vs 9).  We therefore use the DvDvN implementation as it simultaneously categorizes drum types and separates noise from drums. However, without manual inspection, we cannot confirm the extent of this model's usefulness. 

We create the mixed ear model by combining our encoder with the ExtraTrees classifier. As sounds are generated with the virtual synthesizer, the encoding and envelope features will be extracted using the encoder and sent to the ExtraTrees DvDvN classifier. In the upcoming novel generations Section, we will present the results of a two person survey where the accuracy of the two-ear model is analyzed given 500 synthetic noise sounds categorized as drums. 





\end{document}