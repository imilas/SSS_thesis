% Allow relative paths in included subfiles that are compiled separately
% See https://tex.stackexchange.com/questions/153312/
\providecommand{\main}{..}
\documentclass[\main/thesis.tex]{subfiles}
\externaldocument{}



\begin{document}

\chapter{Introduction}


\section{Our Goals}
Digital recordings of novel, one-shot\footnote{A single hit on the drum that captures its capabilities} drum sounds are not easy or cheap to find. New drum sounds can be obtained via live recordings, layering existing drum sounds, and meticulous combining of sound engineering techniques. The goal of this thesis is to program virtual synthesizers to generate sounds that are suitable substitutes for conventional drum sounds. In addition, given that the generated sound is percussive, it should be grouped with the drum sounds that it resembles.

% These results indicate that the majority of sounds produced from our system are suitable substitutes for drum sounds. Among these percussive sounds, we found a moderate to high degree of agreement between the  automatically assigned drum labels and blinded, manual hearing tests.
% \subsection{Drums and Percussion}
% Drums are a subcategory of percussive instruments. Almost any object which is struck to produce noise can be referred to as a drum~\cite{latham2002oxford}. Under strict definitions, the terms can be used interchangeably~\cite{latham2002oxford}. We adapt this strict definition as our work is not concerned the delineation between the two overlapping concepts, but with creation of synthetic sounds which can be used by digital sound artists as replacements for organic drums or percussion. We often use the terms \enquote{drum} and \enquote{percussion} interchangeably.

\section{Why Does the World Need More Drums?}
Percussive sounds such as kick drums and snare drums are often created by striking percussive instruments in various ways~\cite{barry2005drum}. These sounds are commonly used to create rhythm in musical compositions~\cite{needham1967percussion}. In this work, we are not concerned with the nuances of drums versus percussive sounds, as a result, we often use the terms \enquote{drum} and \enquote{percussion} interchangeably.

A common approach to the creation of drum tracks for digital music is to combine short recordings of drums and other percussive elements in order to create a virtual drum-kit. This approach frees artists from the need to obtain and store real life instruments while enabling endless combinatorial possibilities. However, by relying on recordings of \enquote{real life} drum sounds, we are limited by what instruments exist in the real world and whether or not we have access to clean, one-shot recordings. We believe that virtual sound generation can alleviate these material limitations. 
 
\section{What Solution Do We Propose?}
\label{sec:sol_propose}
We want to virtually create novel drum sounds. We need a generative source that produces audio based on some instructions. Additionally, we need a method of evaluation to help us determine which sounds resemble drums and are worth keeping. We also would like to know what instructions, or programs, caused our audio source to make the sounds we liked, so we can modify and experiment with these programs. In short, our approach requires virtual \textit{generation} and \textit{evaluation} of digital sounds.   

% \section{What Tools Are at Our Disposal?}
% \label{sec_tools_disposal}
% We can conceptualize utilizing digital sound synthesis in various ways. An ideal scenario would be a graphical, 3D simulation of virtual drums where we can modulate parameters such as frame, skin and the type of impact before rendering the sound of the simulation. Unfortunately, such 3D systems are not yet practical \cite{langlois2016toward}. State of the art generative networks utilized in works by Ramires et al. \cite{ramires2020neural} and Aouameur et al.\cite{aouameur2019neural} 
% are another promising avenue, but the computation costs and data requirements remain high. Here, we work towards tractable methods that utilize simple signal generation tools and machine learning techniques to rapidly generate new sounds and evaluate their desirability. Reduction of the examples necessary for effective learning is also a priority.
% Before we further cover details of the project, it's important to give a quick introduction to the basics of computerized audio generation and evaluation that were made use of in this project.
% \subsection{How Can A Computer Make Sounds?}
% \label{sec:computer_make_sound}
% Most people can experience some\footnote{Human audible range is approximately 30Hz to 15000KHz~\cite{zwicker1961subdivision}} of the vibrations in their environments through sound. A computer can make a wide variety of audible vibrations in a number of ways: We can use it to record real life audio and play it back. We can hit it with drumsticks. We can also program it to make noise using software, usually by emulating the physical characteristics of basic waveforms by evaluating the output of periodic functions such as sines and cosines. Evaluation of basic periodic functions is the simplest form of software audio generation~\cite{mitchell2009basicsynthChap5}.

% \enquote{Dialing tones} (DTMF) and emergency vehicle sirens are common examples of simple, digitally generated sounds. A simple method of making digital audio is feeding a range of equally spaced values between 0 and $2\pi$ to a sinusoidal function at a rate that's audible to human ears. 

% This generative system is called an \textit{oscillator} and its output is a waveform which can be sent to a computer's digital-to-analog conversion system to create an audible signal. The combination and modification of these pure tones are the building blocks of digital signal processing (DSP).


% \begin{figure*}[h]
% \label{fig_example_sine}
% \centering
% \includegraphics[width=0.45\linewidth,angle =-90 ]{images/periodic_function.png}
% \caption{A computer can simulate waveforms by utilizing periodic functions. Digital waveforms are discrete approximations of analogue waves. Figure adapted from Mitchell~\cite{mitchell2009basicsynthChap5} }
% \end{figure*}


%  Most organic sounds---sounds from the natural environment, and its flora and fauna---are much more complex than the output of a single oscillator. In such cases, their digital approximation requires a combination of multiple (perhaps thousands) of pure tones. However, with careful programming, DSP techniques can be used to replicate almost any sound, which is why they have driven commercial digital synthesizers for over half a century \cite{jenkins2019analog} and continue to remain a popular method of audio synthesis. Another advantage to using DSP for sound generation is that the synthesizers we build using these functions are tractable: the output is reproducible given the same parameters. This makes the evaluation of a set of inputs (or parameters) to our synthesizers simpler when compared to the evaluation of synthesizers that utilize probabilistic models.
 
% In the next chapters we will discuss our work towards a system for automatic \textit{generation of sounds}, capable of rapid generation of short (under 1 second) audio clips. Our approach entailed the implementation of a virtual sound synthesizer that can take a set of instructions, or \textit{programs}, as input and generate the corresponding audio. A by-product of this approach is that the sounds generated may sound extremely in-organic, yet perfectly usable by more experimental artists, adding a desirable "novel" factor. 



% \subsection{How Can a Computer Evaluate Sounds?}
% Automatic evaluation of sounds is an essential component of our work. A thorough manual evaluation of outputs is not possible when hundreds can be created in a second. How can a computer help us evaluate sounds?

% Suppose a person is given a set of recordings of solo musical instruments being played by various skill levels and asked to categorize them however they please. There is a long list of features that the person could use for unsupervised categorization: the length of audio recordings, the skill of the player, timbre, rhythm, and so on. 

% Sounds can be represented digitally as a series of numbers (discussed further in Section~\ref{sec_sampling_rates}). Given this representation, computers can be programmed to quickly extract the majority of these feature types for us. With shorter audio clips (such as the drum sounds we're interested in), personal subjectivity might play a smaller role in describing sounds, making the categorization task easier.

% In our work, we find simple features such as frequency content (high pitch vs low pitch), length, and envelope (change in loudness, how fast the sounds reaches its maximum loudness and fades away) to be powerful for the categorization of drum sounds. We will discuss our algorithms for extraction of these and other features in Chapter~\ref{implementation}. We will also discuss how these extracted features were used to train models that can automatically categorize new sounds.  
\section{What Is Our Methodology?}
\label{sec_methodology}
This is our plan for a generative system which can imitate sounds: a virtual synthesizer continually receives random programs and creates the corresponding sounds, while a virtual ear evaluates and assigns a score to each sound, this score is then used for the separation of undesired outputs from desired ones. This approach assumes that a fraction of the randomly generated sounds can be substituted for percussion and that the virtual ear will assign higher evaluation scores to this subset. While only random search was used in this work, our implementation allows for the future integration of heuristic search algorithms such that the parameters of the synthesizer can be selected based on the previously observed evaluations. Considering these requirements, we found the proper implementation of 2 major components to be crucial:

\begin{itemize}
    \item \textit{Virtual Synthesizer}: A flexible, deterministic, and tractable synthesizer that can create audio. 
    \item \textit{Virtual Ear}: A classifier that returns an evaluation of an audio sample; estimating an audio sample's fulfillment of a musician's requirements. The virtual ear's evaluation guides the generation process towards a desired path, making it a crucial component of our pipeline. 
\end{itemize}

We approached this implementation with modularity and parallelizability in mind. This allows each component to be debugged, modified, and improved without requiring modifications in other components while increasing scalability and speed of experiments. 


While the main focus of this project is the generation of novel percussive sounds, our methodology indicates promising results in regard to the creation of new presets for any virtual synth without a-priori knowledge of its parameters. We also demonstrate the viability of virtual synthesizers based on Digital Signal Processing (DSP) methods for fast, unsupervised creation of novel audio. 


% Instead of learning weights and parameters in an audio-generation neural network, we wish to generate, search, and tune synthesizers to produce percussion sounds. Discussed in more detail in Section \ref{related}. 


\section{Thesis Statement and Contributions}
\label{sec:thesis_statement}
 The focus of this work is the creation of new audio samples for use by music producers. We propose a virtual approach to creation of digital drum-kits, by random and rapid programming of virtual synthesizers and use of virtual listeners for extraction and organization of synthesizer sounds resembling drums. We seek to implement a system of virtual sound generation where the majority of sounds generated can be used in digital music compositions as a replacement for organic recordings of at least one type of percussive instrument.\\
  
  The principal contributions of this dissertation are:
\begin{itemize}
  \item We introduce a framework for rapid, high quality sound generation with virtual synthesizers using simple digital signal processing techniques. 
  \item For training of virtual listener models, we curated a dataset of free percussive sounds which is made available to future researchers~\cite{salimihindle_drums}. 
  \item We train several machine learning models that can distinguish drum sounds from non-drum sounds and categorize the type of drum sounds. We measure and verify the accuracy of these models using organic drum sounds. 
  \item We combine our methods of synthesis and sound categorization to create systems capable of automatic generation of drum sounds of various categories and create synthesizer programs which can be further modified by sound engineers.
  \item We demonstrate the viability of our approach for the creation of digital percussive sounds by conducting manual hearing tests. Based on our blinded hearing tests, most of the outputs of our generative system of sounds are suitable replacements for drums of various categories. 
\end{itemize}



% The goal here was to create an unsupervised system that generates discrete sounds which are suitable substitutes for one-shot recordings of organic drums. In addition, given that a sound is percussive, we require the accurate categorization of drum-types. These definitions can be subjective and/or context dependent: \enquote{suitable substitute} varies by producer and type of music, and we can only define \enquote{one-shot recordings of organic drums} via examples. These definitions necessitate that the agreement between the outputs of the system and surveyed individuals is measured in a blinded way. 



\end{document}