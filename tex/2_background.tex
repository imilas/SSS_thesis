% Allow relative paths in included subfiles that are compiled separately
% See https://tex.stackexchange.com/questions/153312/

\providecommand{\main}{..}
\documentclass[\main/thesis.tex]{subfiles}
\externaldocument{}

% \iffalse
% - Capturing and playing sound, digitally
% - Representation (single frame vs multi(spectrum))
% - Creating sound
%     - Sound can be created functionally
%     - sound can be created piece by piece 
% - a quick history/summary of computer music technology (the parts of it that are relevant to our project)
%     - Define vsts, synth parameters, filters, eqs since some they're involved in some of the previous works.
% - Past work involving heuristic search and digital synths
% - Past work involving generative neural nets
% - Our work and how it's different or extends past work 
%     - How it can be replicated
% \fi

% todos:
% properly use amplitude and magnitude and power
% haven't described timbre 
% make reconstruction wave include multiple waveforms and how they go missing if sampling rate too slow
% haven't described nyquist frequency
    
\begin{document}

\chapter{Background and Related Work}
\label{section:background}
This chapter aims to provide a background for the subsequent chapters by providing a quick overview of four important topics:\\ 
\begin{enumerate}[label=(\roman*)]
\item Digital sound, its features, and concepts that have been fundamental to our work.
\item Common digital synthesis techniques.
\item The applications of artificial neural networks (ANNs) for feature extraction and sound production. Although similar results can be yielded from either approach, we distinguish ANN based techniques from tradition digital signal processing (DSP).
\item Related works and their relative similarities and distinctions.
\end{enumerate}
\section{Digital Audio: Sound from Numbers}
\label{sec:digital_audio}
Sound is the result of a series of physical events. Most of what we hear is the product of physical disturbances, causing vibrations in our mutually shared, immersive mediums. Sound waves are vibrations traveling through air as part of an expanding, spherical wave front, exponentially losing intensity as they travel away from the source~\cite{cook1999chap4}. 
\\

% [graph describing wave amplitude, phase and frequency here]
A sound wave can be viewed as the result of a function which governs amplitude through time, where time and amplitude exist in continuous dimensions. Waves can be approximated via a series of samples, associating time steps to a discrete range of amplitude values. 
Given a wave generation method, computers can make sound by sending a series of discrete values to a digital to analogue converter(DAC), which in turn can
create vibrations within a speaker.  \textit{Digital synthesis} of audio is the process of creating these discrete values. 
\\

\subsection{Sampling Rates and Quality of Digital Audio}
\label{sec_sampling_rates}
In 1963, Mathews wrote on the potential and utility of computers as digital instruments~\cite{mathews1963digital}. He presented a snapshot of digital audio technology of his time and made predictions on what would be possible in the future~\cite{mathews1963digital}. This work by Mathews makes the robust foundations of digital signal processing apparent. Many of the techniques described by Mathews have not only remained popular and relatively unchanged, but also benefited from the increase of computational power throughout the decades~\cite{mack2011fifty,smith1991viewpoints}. For instance, Mathews described a general method for computers to capture and internally represent audio: by discretely sampling continuous pressure waves of sound and recreating the sound from the recorded numbers. 

Sound can take on the physical characteristics of a waveform~\cite{cook1999chap4}. Imagine that the curve shown in Figure~\ref{fig_sampling_rate} is representative of a sound wave we would like to digitally capture. A microphone enables a computer equipped with an analogue-to-digital converter chip to record samples of information about sounds waves at fixed intervals. This means that the original, analogue waveform is now recorded as a discrete, digital signal. Each recorded sample would represent the amplitude of the wave at a time-step. The more packets of information we get, the better our digital recreation of the original sound.

In this context, \textit{sampling rate}, is an important feature of digital sound, referring to the number of samples per second of audio (measured in hertz, or Hz). Sampling rate is not the only important factor when recording audio as it is important to record with not just speed, but also precision. Assuming perfect sensors, precision is the range of possible values we can assign to each sample. It is determined by bit depth: the number of bits we have to represent the values of each sample.  Today, standard quality audio often refers to sampling rates of 44.1 kHz and 48 kHz and bit depth of 16 (that is, $2^{16}$ discrete values), while \enquote{high quality} audio indicates an increase in bit rate or bit depth~\cite{reiss2016meta}. Although subject to diminishing returns, high quality audio (e.g., 96 kHz/24-bit) may be preferable to most musicians and audio-engineers. In a meta-analysis of digital sound perception, Reiss found a small but statistically significant portion of people are able to discriminate the effect of standard and high quality audio with no prior training, and a dramatically higher detection rate after extensive training~\cite{reiss2016meta}. 


\begin{figure}[tbp]

\centering
\includegraphics[width=1\linewidth,angle =-90 ]{images/periodic_function_decimation.png}
\caption{Inadequate sampling rates can make reconstruction ambiguous. While reconstruction is possible in this case, consider the case when multiple signals with varying frequencies are overlapped, or the case when samples are further apart than 1 wave length.} 
\label{fig_sampling_rate}
\end{figure}


\subsection{Loudness, Amplitudes and Envelopes}
\label{sec:adsr}
Loudness is a subjective description of a sound's intensity or energy levels. It varies based on the complexity of sounds, the frequencies present, and hearing ability of the listener~\cite{fletcher1933loudness,cook1999chap6}. It can only be measured relatively, by establishing a benchmark sound and surveying populations on the relative intensities~\cite{cook1999chap6}. Since loudness and intensity of sound correlate with the amplitude of digital waveforms (the values assigned to the samples), an imperfect but convenient alternative method for inferring the loudness of digital sounds is to compare relative amplitudes. A common function for inferring the loudness of digital signals is to apply the Root Mean Square (RMS) function to its samples~\cite{zwicker1977procedure}. 

Sounds typically vary in intensity as they unfold. This change in intensity is often described by the \textit{envelope} of the sound, particularly in shorter samples. For digital sounds, Mitchell describes the envelope as either of the borders (since samples typically take the range of -1 to 1) that are created by graphing a signal and connecting the local absolute peak values~\cite{mitchell2009basicsynthChap6}. In electronic music production, the envelope is generally defined using 4 features: Attack, Decay, Sustain, and Release (ADSR). Attack describes how quickly the peak loudness is reached. Decay for how quickly the sound drops to sustain level. Sustain is the duration of sustaining intensity (for example, how long a finger is kept on a piano key). Release describes the speed of fading to silence (how fast the sound decays once the piano key is released). We will describe our method of approximating the envelope of sounds in Section~\ref{sec:fourier_transforms}.

Envelopes can be mathematically described and used to shape signals. A common approach in digital sound synthesis is to output all samples at a consistent amplitude and apply an envelope later down the synthesis chain. Digital and analog synthesizers often have built-in ADSR modules to shape the volume of the output and other parameters.  

% [Graph of envelope shaping a signal]
% note: make sure you're using terms signal and waveform properly
\subsection{Frequency, Pitch and Spectrograms}
\textit{Frequency} is used to describe number of repetitions within a time-frame, or how frequently a cycle is repeated. As discussed before, frequency of an audio signal is often measured in unit of \textit{hertz} (cycles per second). Most sounds, particularly those from non-virtual sources, are a combination of multiple different pressure waves with different frequencies and amplitudes. \textit{Pitch}, is a perceptual property that is tied to the frequencies present in a sound. How we perceive and describe the pitch of a sound (e.g., high-pitched vs low-pitched) is heavily dependent on the characteristics (frequency, amplitude, duration, etc) of the waveforms it contains. Some sounds, such as piano keys or pure tones have a discernible pitch. Others, such as \enquote{pink noise} or the sound of rain, do not. Yet another factor to consider is the hearing ability of the subject, which varies between people based on factors such as age, environment, and musical training~\cite{reiss2016meta,alain2007age,newman2012grm7}. 

% We often do not have access to the time-variant systems which create sounds. If we have access to a recording of their outputs (i.e digital sound), we can make approximations about their characteristics. 
Spectrograms are graphs used to depict the duration and amplitude of frequencies present in a sound. To create spectrograms, sound must be decomposed into a set of simpler functions. A common method for the breakdown of complex, time-variant functions is the Fourier transformation and its many variations. One such method is the discrete Fourier transformation (DFT) and the inverse DFT. DFT and inverse DFT can convert digital sound from its time domain representation (sequence of samples) to its frequency domain representation (sets of frequency ranges and their amplitude) and vice versa. We share some examples along with our methodology of creating spectrograms in Section~\ref{sec:fourier_transforms}.\\\\

% [spectrogram figures]

% https://en.wikipedia.org/wiki/Frequency_domain#/media/File:Fourier_transform_time_and_frequency_domains_(small).gif

\section{Digital Audio Synthesis}
\label{sec_digital_synthesis}
The phenomena of sound at intensities we commonly encounter can be described as a product of a \textit{linear system} of functions~\cite{cook1999chap4}. A linear system is a system where the transformation of overlapping inputs is equal to the sum of the separately transformed inputs~\cite{lyons2004understandingChap1,cook1999chap4}. In a linear system $\mathcal{S}$ with valid inputs and outputs $x(i)$ and $y(i)$, if we have:

\begin{equation}
 \mathcal{S}(x(i_1)) \xrightarrow{generates} y(i_1)
\end{equation}
\begin{center}
    and
\end{center}
\begin{equation}
\mathcal{S}(x(i_2)) \xrightarrow{generates}y(i_2)
\end{equation}

The output of the system given both inputs is the sum of the individual outputs, or:
\begin{equation}
 \mathcal{S}(x(i_1)+x(i_2)) \xrightarrow{generates} y(i_1)+y(i_2) 
\end{equation}
\\
This concept has important implications digital audio creation and analysis. Simple tones can be combined to create complex sounds, and complex sounds can be broken down for easier analysis~\cite{lyons2004understandingChap1}. It also allows experiments with simple sine waves to remain relevant in complex sound domains~\cite{cook1999chap4}.

% [graph of addition of two sine waves making a complex sine wave ]
% \\

Various sound synthesis techniques have been developed by treating sound as a sequence of values. Linear systems are commonly used in creation of musical tones, while non-linear systems are used for introduction of distortion and noise where needed. In their taxonomy of digital synthesis techniques, Smith defines four families of algorithms: algorithms that process and modulate existing sounds (e.g., granular synthesis, wavelets), spectral models that aim to create a particular spectrum of sound (e.g., additive, subtractive), physical models which emulate the physics of real instruments, and abstract models (e.g., wave shaping, Karplus-Strong), often used for adding harmonics or distortion to simple sound signals~\cite{smith1991viewpoints}. 

Synthesizers are engines of synthesis that make use of one or more of these techniques for sound generation. Selection of the appropriate synthesis method depends not only on the expectations for the end product, but also the features of the synthesizer itself. Whether in goal oriented tasks such as text-to-speech or in creative endeavors such as ambient-noise generation, it is often desirable to work with systems that are quick, adaptable and tractable. For example, one might desire a text-to-speech system where slight changes to input parameters can introduces slight changes to the speech patterns, utterances, voices, etc. This ability to quickly modify and audition sounds becomes a necessity when the synthesizer is being used as a creative instrument in of itself, rather than an emulator for existing instruments and sounds. 

Often used methods of digital sound generation are \enquote{additive} and \enquote{subtractive} synthesis, umbrella terms for some of the most simple and common methods of digital synthesis~\cite{mitchell2009basicsynthChap1}. 
In additive synthesis, sounds are built as a sum of signals, where signals are outputs of oscillators (periodic wave generators).  In subtractive synthesis, segments of a complex signal are removed until a desired sound is reached. A chain of one or more \textit{digital filters}, which can subtract or reduce frequency ranges, are often used in subtractive synthesis. Digital \textit{low-pass filters} lower the amplitude of signals with frequencies higher than a given \textit{cutoff}, while \textit{high-pass} filters remove lower than threshold frequencies. It is not uncommon for percussive sounds to have noisy, chaotic high frequency content during their short attack period, followed by harmonic low/medium frequencies~\cite{lakatos2000common}.


% \subsection{Analog Synthesizer}
% Talk about analog synthesis briefly, so the word synthesizer feels less abstract...
\subsection{Virtual Synthesizers}
Nearly 5 decades ago, Mathews claimed that any sound can be recreated via a computer by high frequency sampling of pressure waves~\cite{mathews1963digital}. He noted that since \enquote{a very high sampling rate is required...if this process is to be useful musically, programs for generating samples from the parameters of notes must be written}~\cite{mathews1963digital}. The methods of synthesis discussed here are a major component of such programs. With the exception of physical synthesis, modern computers are more than capable of simultaneously running many instances of these algorithms. To further assist with their musical utility, the majority of digital synthesis systems work in tandem with programs such as Musical Instrument Digital Interface (MIDI), which can modulate the parameters of these synthesis methods by modulating information pertaining to ADSR and other note characteristics, often in real time~\cite{moog1986midi}.  

 The rise of Digital Audio Workstations (DAWs)~\cite{leider2004digital} and Virtual Studio Technology (VST) plug-ins~\cite{tanev2013virtual} have rapidly transformed the sonic and material landscape of music production in the recent years. Coupled with this rise in popularity is a vast array of commercial products and services which cater to the need of amateur and professional music producers for unique sounds, often by provision of audio samples; one-shot drum samples, long sustained notes (referred to as pads or textures), and loops (percussive or melodic) are common deliverables. Two notable examples of these commercial services are \textit{loopmasters}\footnote{loopmasters.com} and \textit{splice.com}\footnote{splice.com}. VST plug-ins can emulate analogue synthesizers and effects,  however,  due to their (often) complex interface, some producers may find VST plugins daunting to work with from scratch. In many cases, VST plug-in vendors or unaffiliated enthusiasts sell additional presets for these plugins, targeted towards producers who do not have the time or interest in creating their own. The flexibility of the VST technology allows producers to modify these presets until their desired sound is reached.
 
\section{Neural Networks And Sound}
\label{bg:NN}
In Section~\ref{sec:digital_audio}, we defined digital synthesis as the \enquote{process of generating discrete values which approximate sound waves}. We also established that manual generation of these values at high sampling rates is not practical; a problem which has motivated a wide variety of synthesis techniques which aim to create signals within a linear (or mostly linear) system. The recent exponential increase in computing power has been coupled with a wide range of research in probabilistic sound generation, mainly via generative neural networks. 
% deterministically, if the input parameters to the system do not change, the output of the system will remain the same \footnote{talk about noise generation methods and seeding}.

Artificial Neural networks (ANNs) map inputs to outputs via a large network of parameters and activation functions. Given the right network shape and parameter weights, they can approximate a large set of functions~\cite{cybenko1989approximation,cardaliaguet1992approximation}. ANNs are often deployed when we do not have access to the system of functions which guide a process, but a mapped set of inputs and the corresponding outputs are available. Given this set, the parameters of a neural network can be tuned for approximating the effect of the system on any valid input. The architecture of the neural network (e.g., number of layers, connections, activation functions) is often selected via trial and error ~\cite{bergstra2012random,bergstra2011algorithms,ba2013adaptive}. By definition, these approximations will never be more accurate than the system that is being approximated. 

Since their emergence in the 1950's, research on ANNs has gone through several eras of stunted growth~\cite{basheer2000artificial,anderson1988neurocomputing}.  In the last decade, the increase in the affordability of high performance graphic cards has been coupled with a major resurgence of interest for ANNs and the emergence of a number of domain specific variations of the traditional ANN architectures (see Section~\ref{related}). 

Generative neural networks (GNNs) are utilized for the completion of sequences of values; often by taking an incomplete sequence as input and outputting the most likely value for the next step. The WaveNet architecture introduced in 2016 is considered a seminal breakthrough in the usage ANNs for sound synthesis~\cite{oord2016wavenet} by surpassing state of the art speech synthesis techniques, which create outputs with the the combination of previously recorded audio snippets~\cite{schwarz2007corpus}. When trained on a large corpus of audio samples, GNNs such WaveNet can learn the \enquote{predictive distribution for each
audio sample conditioned on all previous ones}~\cite{oord2016wavenet}. Once this distribution is learned, it can be used to create sounds 1 sample at a time, a slow process, as Mathews predicted~\cite{mathews1963digital}. \\
% Parallel WaveGAN has only 1.44 M parameters and can generate 24 kHz speech waveform 28.68 times faster than real-time on a single GPU environment. Perceptual listening test results verify that our proposed method achieves 4.16 mean opinion score within a Transformer-based text-to-speech framework, which is comparative to the best distillation-based Parallel WaveNet system.


% \section{Goal Oriented Novel Sound Generation}


% \subsection{Tool Selection}
% We define \enquote{goal oriented novel sound generation} as any work that seeks to implement a system that is capable of generating novel sounds with a desired characteristic. Based on our review of relevant works, we believe that \enquote{goal oriented audio generation} necessitates two essential components: A tool-set for the analysis of sound and a tool-set for creation of sound. As a result, we base our work around learning of the distinguishing features of various sound groups (i.e different types of drums) and using the learned features for generation of sound. In Section~\ref{sec_methodology} we introduced these components as the virtual ear and virtual synthesizer. 

%  Thus far we have discussed various techniques at our disposal for the implementation of a virtual ear and synthesizers. In the upcoming Sections, we will discuss the implementations and the subsequent results. Here, we will review a variety of works which fall under our definition of \enquote{goal oriented novel sound generation}. Particularly, we are interested in a discussion of goals, chosen methods for the feature extraction, and chosen methods of sound synthesis. 

\subsection{Related Works}
\label{related}

\begin{center}
\begin{table}[]
% \resizebox{\linewidth}{!}{
\begin{tabular}{||c c c c||} 
\hline
work & feature extraction & synthesis & specilization  \\
	\hline
Oord et al.~\cite{oord2016wavenet} & CNN & CNN &Speech \\ 
	\hline
Yamamoto et a.l~\cite{yamamoto2020parallel} & GAN & GAN&Speech  \\ 
	\hline
Aouameur et al.~\cite{aouameur2019neural} & Latent layer& Decoding of Latent Layers & Percussion  \\ 
	\hline
Ramires et al.~\cite{ramires2020neural} & Latent layer & FeedForward Network & Percussion  \\ 
	\hline
Yee-King et al.~\cite{yee2018automatic} & LSTM on Paremters & DSP & Synth Pads  \\ 
	\hline
\end{tabular}
%}
\caption{Quick reference for related works}
\end{table}
\end{center}

ANN or DSP approaches can be taken towards the implementation of a virtual ear and a virtual synthesizer. The recent development of ANN frameworks has led to works which have utilized ANNs for both components~\cite{oord2016wavenet,yamamoto2020parallel,ramires2020neural}. Also common are works which have leveraged a mixture of both approaches, often by utilization of ANNs for the virtual ear and DSP methods for synthesis~\cite{aouameur2019neural,yee2018automatic}.

Many deep neural network models have been proposed and utilized for the purpose of signal generation in recent years. WaveGans and WaveNet have been subject to significant improvements and experiments since their proposal~\cite{nsynth2017,yamamoto2020parallel,oord2017parallel}. Specifically for the generation of percussive sounds, a recent work by Aoumaeur et al.~\cite{aouameur2019neural} utilizes variational AutoEncoders (VAE's) for generation of drum sound spectrograms, which are then converted to sound using a Multi-head CNN model~\cite{aouameur2019neural}. Another recent work by Ramires et al.~\cite{ramires2020neural} also uses neural networks for this purpose, where a feedforward neural network capable of creating sounds is guided by a small number of parameters which represent the producer's desired characteristics for a drum sound. 

Automatic programming of virtual synthesizers has also been a topic of interest. Genetic Algorithms have long been utilized for the generation of new sounds with various sound-engines~\cite{johnson1999exploring,dahlstedt2001creating,hornermachinetongues,macret2012automatic}. More recent work by Yee-King et al.~\cite{yee2018automatic} used Long Short-Term Memory (LSTM) models and genetic algorithms to find the exact parameters used to create a group of sounds. The sounds approximated were made by the same virtual synthesizer, not an external source; making the eventual replication certain even with random search. In addition, the work by Yee-King et al.~\cite{yee2018automatic} is generally more focused on pads and textures rather than drums, and feature matching appears to not be concerned with the envelope of the sounds but rather the frequency content within arbitrary time windows. Yet another recent work by Esling et al. used a large dataset of over 10,000 presets for a commercial VST synthesizer to learn a latent parameter space which can be sampled for creation of new programs for audio synthesizers~\cite{esling2019universal}. This bespoke latent space requires large amounts of synthesizer programs for the initial training, and cannot be used for other virtual synthesizers. In the work presented here, we take a different approach to automatic programming of synthesizers by aiming for rapid approximation of percussion sounds with no previous knowledge about the sonic capabilities of our virtual synthesizer and exploring the actual parameter space rather than its latent representation. Unlike previous related works, no prior examples of audio made by the synthesizer nor examples of manually generated programs are needed for our approach. Any synthesizer can be integrated into our system so long as \begin {enumerate*} [label=\itshape\alph*\upshape)]
  \item its parameters are known,
  \item the system can randomly modify its parameters, and
  \item The system can render and extract the sound output for virtual listening tests.
\end{enumerate*}  
These requirements are not strict as VST synthesizers meet these expectations by design, giving our project a large scope of applicability. 
\end{document}
