% Allow relative paths in included subfiles that are compiled separately
% See https://tex.stackexchange.com/questions/153312/
\providecommand{\main}{..}
\documentclass[\main/thesis.tex]{subfiles}

\begin{document}

\begin{abstract} 
% We create systems that automatically generate and categorize drum synthesizer programs, and measure the success of these systems by conducting manual listening tests of the generated sounds. Can we create systems of drum sound generation where the majority of outputs sound like drums to human listeners?

This work focuses on the virtual generation of short percussive samples which can be used by electronic music artists in their compositions. Although recent advancements in digital synthesis, heuristic search, and neural networks have been utilized for the generation of a variety of sounds, the lack of access to large audio datasets, the problem of open set recognition, and high computational costs persist as barriers towards the expansion of digital sound libraries using these techniques.
We present our approach towards the automatic generation of synthesizer programs which mimic one-shot percussive sounds. This work documents the implementation of an automatic system for generation of virtual percussive synthesizer programs using classical signal processing and machine learning. This system relies on virtual ears to find synthesizer programs which mimic percussive sounds, and to further categorize these programs into a number of common drum types. We demonstrate promising results in both detection and categorization of percussive sounds by representation of digital audio through Fourier transformations and autoencoder embeddings. Manual listening tests of the generated sounds indicate that the system can successfully generate drum synthesizers and categorize drum sounds. To facilitate future research, we share our curated datasets of free percussive sounds. These datasets can also be used for the replication of our work.
\end{abstract}



% \begin{abstract} CSMC Abstract\\
% Can we generate drum
% We present an approach for the automatic generation of synthesizer programs for one-shot percussive sounds. 
% Recent advancements in digital synthesis, heuristic search, and neural networks can be utilized for sound generation. 
% Yet the need for data, the problem of open set recognition, and high computational costs persist as barriers towards the expansion of sound libraries using these techniques. 
% We generate quick, scalable, percussion synthesizers using classical signal processing. 
% We train drum classifiers to find and classify synthesizer programs that mimic percussive sounds. 
% We use features from Fourier transformations and autoencoder embeddings to train machine learning classifiers.
% Manual listening tests of the generated sounds demonstrates the system can successfully generate drum synthesizers and categorize drum sounds.
% To facilitate future research, we share our curated dataset of free percussive sounds.

% \end{abstract}


\end{document}