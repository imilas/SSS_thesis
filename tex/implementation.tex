% Allow relative paths in included subfiles that are compiled separately
% See https://tex.stackexchange.com/questions/153312/

\providecommand{\main}{..}
\documentclass[\main/thesis.tex]{subfiles}

\onlyinsubfile{\zexternaldocument*{\main/tex/introduction}}

\begin{document}

\chapter{Implementation}

% \section{Approach}
% \subsection{ANN approaches}
% Numerous deep, neural network models have been proposed and utilized for the purpose of signal generation in recent years. WaveGans and WaveNet have been subject to significant improvements and experiments since their proposal \cite{nsynth2017,yamamoto2020parallel,oord2017parallel}. Even more recently Variational AutoEncoders (VAE's) have been utilized for generation of short percussive samples \cite{aouameur2019neural,ramires2019timbfeat}. In this work however, we opt to use digital signal processing methods to create a virtual synthesizer for the generation of audio signals as it provides several unique advantages:
% \begin{enumerate}[label=\roman*]
%   \item Fast, offline rendering of audio with no reliance on GPU: Currently not possible with state of the art models such as parallel WaveGan \cite{yamamoto2019parallel} and parallel WaveNet \cite{oord2017parallel}. 
%   \item Rendering at high sampling rates: Performance speed being a common issue, the standard sampling rate in most audio generation work utilizing neural networks appears to be under 24 khz \cite{yamamoto2019parallel,oord2017parallel,aouameur2019neural,ramires2019timbfeat}. However, a significant number of untrained human ears can detect a change in quality of audio between sampling rates of 192 khz and the industry standard of 44.1 khz \cite{reiss2016meta} with a dramatic increase in quality detection after training. Therefore we can safely assume that most producers would prefer their audio samples to have sampling rates of 44.1 khz or higher. In this work, we fix our sampling rate to the 48 khz standard. 
%   \item Neural networks are viewed as unexplainable black box solutions \cite{basheer2000artificial}. Some models such as VAE's can learn an underlying latent space of parameters and capture the distinguishing features of the different labels in a dataset. However, these spaces are learned in an unsupervised manner and must be manually analysed, perhaps extensively, before they can be understood \cite{esling2018generative}. The use of a virtual synthesizer for audio generation makes our parameters readily understandable and easily modifiable. \\
% \end{enumerate}

% Automatic programming of virtual synthesizers has also been a topic of interest. In early 2000s, Interactive Genetic Algorithms (IGA's) were utilized for the generation of new sounds with various sound-engines \cite{johnson1999exploring,dahlstedt2001creating}. More recent work by Yee-King et al. \cite{yee2018automatic} used Long short-Term Memory (LSTM) models and genetic algorithms to find the exact parameters used to create a group of sounds. The sounds approximated were made by the same virtual synthesizer, not an external source; making the eventual replication certain even with random search. Since this work appears more focused on pads and textures rather than drums, feature matching appears to not be concerned with the envelope of the sounds but rather the frequency content within arbitrary time windows. Yet another recent, impressive work by Esling et al. used a large dataset of over 10,000 presets for a commercial VST synthesizer to learn a latent parameter space which can be sampled for creation of new audio \cite{esling2019universal}. As stated before, our work explores the rapid approximation of percussion sounds with no previous knowledge about the sonic capabilities of our virtual synthesizer, exploring the actual parameter space rather than its latent representation.
\label{impl}
\section{Our Goal}
Our goal is to create a system for the generation of novel drum sounds. In this chapter, we will discuss the two main components of our system: The \textit{virtual synthesizer} and the \textit{virtual ear}. A quick outline of our generative pipeline is this: Random programs will be sent to the virtual synthesizer, modifying its parameters. As the virtual synthesizer creates sounds based on these programs, the virtual ear will listen to the sounds and determine if they should be categorized as drums. If so, which category of drum do they belong to. 

Here we discuss in more detail the implementations and interactions of these two components. Later on in the chapter, we will discuss our dataset and code which can be used for replication of our project. 
\section{Virtual Synthesizer}
\label{vs}
We require a synthesizer of sound which is capable of rapidly receiving programs and rendering the corresponding sound offline. We anticipate that any synthesizer with such capabilities can be integrated into our system; Here, we do not prioritize perfect replication of organic sounds as unusual synthesis methods will likely lead to novel sound generation. As discussed in section~\ref{sec_digital_synthesis}, we opted for a set of classical DSP methods to build our synthesizer. For our project, we used the python based Pippi library for sound generation\footnote{https://github.com/luvsound/pippi}. This library uses a C back-end\footnote{https://github.com/PaulBatchelor/Soundpipe} and focuses on fast offline generation of audio signals.

Our virtual synthesizer contains a set of one or more submodules. These submodules have identical sets of parameters, but widely different outputs can be achieved depending on the values assigned. The set of parameters available to each submodule is highlighted in table~\ref{table:submodule_params}. The sonic output of the virtual synthesizer is the normalized addition of the sonic output of its submodules. Our implementation of a synthesizer can have any number of submodules. The parameters that dictate the output signal of each submodule as well as the range of values each parameter can take are shown in table~\ref{table:submodule_params}. We call the number of submodules in each virtual synthesizer the \textit{stack size}. We call the sets of parameter values that characterize a synthesizer's submodules a \textit{program} (analogous to a preset for a VST).  

As we are interested in short, one-shot percussive sounds, each virtual synthesizer program will generate a 1 second piece of audio. This 1 second limit is over twice the length of the average one-shot drum sample within our database. Each submodule can make an audio signal with the length of 0.1-1 second, and play it at any point within the 1 second rendering time (but the entire sound must fit within the second, that is, a 0.5 second sound cannot begin playing after 0.5 seconds within rendering time frame). If the synthesizer has a stack size of more than 1 the audio signals from each submodule are overlapped and the total amplitude is normalized.

The ADSR parameters shape the amplitude of the signal. Prior to being applied to the signal, each of these parameters is assigned an integer value in the range of 0-3, and normalized relative to the others such that \[ A_{norm} + D_{norm} + S_{norm} + R_{norm} = 1 \] \\ 
Where each value $v_{norm}$ in the $\{A_{norm}, D_{norm},S_{norm},R_{norm}\} $ set is normalized such that:
\begin{align*}
\text{for each $v$ $\epsilon$ \{A,D,S,R\}} \\
v_{norm} = \dfrac{v}{A + D + S + R}
\end{align*}

Considering drum sounds such as kick or bass drums where the pitch of the sound can quickly slide from high to low, each submodule can slide between 4 different pitches in the 1 second time frame. Each pitch value is a midi note with a frequency and length value. Each submodule accepts a list of 5 consecutive possible pitch values. The submodule will play each note in the list consecutively after normalizing the length values. The pitch notes are played in a portmanteau fashion such that there is no audible gap. This normalization of length values is similar to that of the ADSR values. Considering this setup, we can predict that in order to generate kick drums using these parameters, the list of notes must vary from high frequencies to low; While other drum types such as hats or shakers require all pitch values to remain static, such that there is not pitch bending. 


The OSC type will determine the shape of the signal. We limited this parameter to three fundamental wave forms: sine waves, square waves and saw waves. We also allow the creation of noise signals, which can imitate timbral characteristics of higher pitched drum samples at a very low computation cost (relative to the addition of thousands of sine waves at various frequencies). If the IsNoise boolean is set to true, the OSC type parameter loses importance as the OSC type will simply be used for the generation of noise via random wavetable transformations. Before the filter and ADSR envelope take affect, the generated noise will have similar characteristics to white noise. 

\begin{table}[h!]
\centering
\resizebox{\columnwidth}{!}{\begin{tabular}{ |c|c|c| } 
\hline
Parameters & Value Range & notes and constraints\\
\hline \hline
Attack & 0-3 & A-D-S-R values relative\\
Decay & 0-3 & relative to A-S-R\\
Sustain & 0-3 & relative to A-D-R\\
Release & 0-3 & relative to A-D-S\\
OSC type & sine,square,saw & tone type\\
IsNoise & boolean & whether to generate noise\\
Length & 0-1 second & - \\
StartTime & 0-1 second & Length+Start$<$1\\
Amplitude & 0.1-1 & 1 = max amplitude\\
Pitches(notes) & list of pitches &  range of C0(16.35hz) to B9 \\
HP filter Cuttoff & 0-20000 & -\\
LP filter Cuttoff & 20000-HP & never lower than HP cutoff\\
Filter Order & 4,8,16 & butterworth filter order \\
\hline
\end{tabular}}
\caption{Synthesizer submodule Parameters. Despite the simplicity of the parameters and our efforts at constraining the ranges, the number of parameters that can be randomly chosen for each submodule is in the order of $10^{15}$ }
\label{table:submodule_params}
\end{table}


\section{The Ear}

What we refer to as an "ear" is any method (such as machine listening) of evaluating a piece of audio, capable of "listening" to a piece of audio and giving it a score (or a list of scores) based on how well it satisfies certain criteria. In this work we are mainly focused on percussive generation, therefore, what we require from the ear is to give us probabilities of an audio sample belonging to various categories. As our synthesizer outputs are close to deterministic for all programs, the evaluations of the ear would allow us to associate scores of the sound with the program that generated it. In section~\ref{gens} we discuss how these scores can be used to navigate synthesis towards parameters which give us the desired sonic output.

 With our dataset of labeled drum sounds, discussed in~\ref{data}, we can confidently categorized unlabeled drum sounds given that we know they are drum sounds. However, a major hurdle towards the implementation of a "drum from non-drum" recognizing ear is that the set of sounds that are not percussive is infinite. 

In our case, drum groups are an example of closed sets, since we believe that a sufficiently large sample pack can effectively describe common drum categories. However, effective representation of all possible non-drum sounds is not attainable via examples alone. We hypothesize that the implementation of our virtual ear falls outside the traditional categorization problem and within the "Open Set Recognition" (OSR) domain. 

Traditional classification tasks often make the assumption the data points used for training the model and future unlabeled data will emerge from the same system of processes~\cite{geng2020recent,mundt2019open}. This assumptions requires that sufficient positive examples of all possible classes exist and are trained on. Works toward the implementations of GANs have documented scenarios in which networks will assign high categorization probabilities to nonsensical, out of context data which should be rejected rather than categorized~\cite{geng2020recent,mundt2019open,hassen2020learning}.  

Once a sound is generated and passed onto the ear, we expect the virtual ear to facilitate our actions in response to two important questions: 
\emph{
\begin{quote}
\text{EQ.1 Could the sound could be used as a drum?}\label{EQ1}
\\
\text{EQ.2 If it does sound like a drum, what type of drum should it be?}\label{EQ2}
\end{quote}
    }


The second question can be answered rather easily and in various ways, as later discussed. However, the former decision requires knowledge of what drums \textbf{do not} sound like, or knowledge of an infinitely large set, which cannot be fully represented via examples. It also means that the source of sounds used in training the model (organic drum sounds) will be fundamentally different from the source of unlabeled sounds we wish to categorize (noise from a synthesizer). We attribute the relative ease of \hypref{EQ2} relative to~\hypref{EQ1} to be reflective of the OSR problem. 

We are not looking for the perfect imitation of organic drums using a synthesizer. We seek to imitate drums using a synthesizer and even \textbf{prefer} for its generations to retain novel, unusual characteristics. The task assigned to the virtual ear is the rejection of "noise" from the virtual synthesizer and the rare acceptance of drum-like sounds. By definition, we cannot anticipate what these novel drums will sound like. Considering the limitations of learning by example, and since raw data can clearly separate organic drums from virtual synthetic noise, we believe that transformation and filtering of raw sound data is a critical step of the listening process; A step that will assist with the extraction of the fundamental characteristics that we are seeking and prevent the rejection of novel sounds.

As our synthesizer rapidly produces sounds, the virtual ear will listen to the sounds and accepts those that satisfy some fundamental characteristics of drums. How we characterize this description is critical as it allows novel sounds to be accepted as part of the drum group despite their anomalies. We approached this problem by the implementation of various feature extraction methods as well as various models which use these features towards a solution for ~\hypref{EQ2} and~\hypref{EQ1}.

We cover the implementation of these virtual ear models in two sections: \emph{two phased ears} (TPEs) and \emph{embedding ears} (EEs). As we will discuss in the next sections, TPEs are a combination different models for each of ~\hypref{EQ2} and~\hypref{EQ1}; With the features utilized by these models being manually defined. EEs use a highly compressed, automatically encoded representation of sound to give simultaneous answers to both questions.

\subsection{Two Phased Ears}
\label{sec:ear}
\subsubsection{Feature Extraction}


   % AH: is the virtual synthesizer a synth or a synthesizer that makes synthesizer
  % AH: you should make it 1 instance of a generated synth 
We only consider 1 second long sounds made with a single stacked synth, yet the set of possible synthesizer parameters (and we assume, the sound generated by the synth given these parameters) is extremely large. In our case, as discussed in Section ~\ref{survey}, the majority of sounds generated by the synth do not resemble percussion.  We determined that there are 2 steps necessary to effectively extract percussive sounds
from the randomly generated sounds produced by virtual synthesizers: 
\begin{enumerate}
   \item  Phase $1$: Binary separation of sounds with percussive features from non-percussive sounds.
   \item Phase $2$: Given confidence that the sound being categorized is percussive, categorizing its type.
\end{enumerate}
We are interested in extracting generalizable, domain agnostic features. In this work we rely entirely on Fast Fourier Transform (FFT) and by extension Short-time Fourier Transforms (STFT) for feature extraction. Using FFT, a signal can be represented by a vector with each index corresponding to a frequency-bin (a range of frequencies too close to be distinguishable) and the value at each index corresponding to the combined-magnitude of the frequencies within the bin. STFT can be employed when a more accurate representation is desired; via the application of the FFT to a sliding time-window on the signal to create a matrix (a list of vectors). This matrix can effectively represent the frequencies present in the signal at each time step, given the right window-length and hop-size (how much the window is shifted at each time-step).
\begin{figure}
\centering
\textbf{Visual Representation of Raw Features}\par\medskip
    \subcaptionbox{Recorded hat sample}{    \includegraphics[width=1\columnwidth]{images/ff1.pdf}
    }
    \subcaptionbox{Randomly generated audio with percussive qualities, resembling a tight snare}{\includegraphics[width=1\columnwidth]{images/ff2.pdf}}
    \subcaptionbox{A randomly generated noise with a percussive envelop but non-percussive frequency features (modulated pitch)}
    { \includegraphics[width=1\columnwidth]{images/ff3.pdf}}
\caption{Graphed representation of features extracted for 3 different samples. Sample $a$ is a recorded hat from our database. sample $b$ is an example of randomly generated noise with percussive qualities that we found suitably similar to a snare sound. Sample $c$ is an example of a randomly generated noise where the spectrum features are necessary for proper classification.}
\label{fig:stackspectrums}
\end{figure}
% AH: If you need space shorten this paragraph to we use STFT which has worked in past and we use spectral-centroids and zero-crossings
Various works have demonstrated effective reconstruction of signals given their STFT~\cite{nawab1983signal,griffin1984signal}. For our purpose, if the original signal can be faithfully reconstructed from its STFT, analysis of the STFT may be relied on as source of fundamental features necessary for audible signal categorization. In the interest of keeping the feature space small and fundamental, other methods of feature extraction such as Spectral-Centroids~\cite{schubert2004spectral} and Zero-Crossing Rates~\cite{gouyon2000use} are not utilized.

Using only the STFT of signals as source of feature extraction we defined 3 transformation functions which we believe to capture important, unique attributes of percussive sounds.  These functions were applied at training time to 1 second audio samples before being sent to a classifier; transforming a signal into a set of features which we hope can capture the necessary information for our categorization task. 


\begin{enumerate}
\item Envelope Transformation: The goal of this feature is to capture the changes in loudness for the duration of the signal. Using STFT we generate a matrix $M_{i \times j}$ with rows $i$ and columns $j$ corresponding to time steps and frequency bins respectively, and with values $v_{i \times j}$ indicating the magnitude of the frequency bin $j$ at each time-step $i$. Information about the envelope of the signal can be extracted by summing the values of $M$ for each time-step (or row $i$), giving us a feature vector $v_i$. This vector is then normalized to the range of 0 to 1. The information contained in this vector is similar to that of a Root-Mean-Square measurement.
\item Frequency Transformation: A static, normalized snap-shot of the the frequencies present within the audio. The calculation of this feature vector is similar to the envelope, but the summation is done along the frequency axis. Another important distinction is that since capturing an adequate frequency resolution is important for this transformation, we utilized shorter hop-sizes and wider windows. A Mel Scale transformation was also applied in hopes that the captured features better represent human perception of frequencies. 
\item Spectrum Transformation: This function is simply a Mel Scaled STFT with its values normalized from 0-1. Since this features is a 2D matrix rather than a vector it captures more information about our signal but requires heavier, more complex computational methods to be utilized. 
\end{enumerate}

\subsubsection{The Models}
Using the described features, we trained several neural network models for Phase 1 and 2 in the Pytorch environment. The task of Phase 1 is to separate drums from not-drums (DrumVsNotDrum, or DVN). The task of Phase 2 is to categorize drums and percussion (DrumVsDrum, or DVD). We kept our feature space small, making it viable for feature selection and model design to be done on a trial and error basis. For all models, accuracy is calculated by prediction of all test dataset labels and the loss function and optimizer are Categorical-CrossEntropy and Adam respectively. Training continues until no reduction in loss and accuracy is observed in 10 epochs.  All activation functions are PReLU:
\begin {enumerate}
\item FC-DVN: Fully connected network trained on Envelope features, reaching 97\% accuracy on our test data for Phase 1. With size of 10x5x10.
\item CNNLSTM-DVN: A combination of CNN and LSTM models, where the CNN model extracts higher level features that are fed temporally to an LSTM cell. This model is trained on spectrum data and reaches 98\% accuracy on our test set. Its structure is the combination of a CNN with 2 output channels and kernel size $(7,3)$; Followed by an LSTM model of hidden size 800 and a fully connected layer of size 20x2.
\item E+F-DVD: A fully connected model trained on a concatenation of envelope and frequency features. Reaching 80\% accuracy for 6-way drum categorization in Phase 2. Size of 50x10x2x6.
\item CNN-DVD: A CNN model trained on Spectrum features. Reaching 82\% accuracy in a 6-way drum categorization in Phase 2. A combination of a CNN model with output channel size of 4, kernel of size of 5, another CNN model with output channel size of 8 and kernel of size 3. Followed by a fully connected network of shape 100x20x6.
\item FC-DVD: Fully connected 3 layer neural net with 78\% accuracy for 6-way drum categorization in Phase 2. Size of 400x200x50.
\end{enumerate}
The parameters are hand-picked and un-tuned. As discussed in section ~\ref{survey}, higher accuracy rates in these models do not translate to higher agreeableness with humans. leading us to believe that model accuracy on test data alone cannot be relied upon when the domain of sounds 
being categorized is switched from original percussion samples to virtual synth sounds.

With our models showing high accuracy on testing data, we combine models in order to increase the efficacy of each phase and address the "open set problem" for our task. For Phase 1 we only determine sounds as percussive if both FC-DVN and CNN-LSTM have determined it as such with over 90\% confidence. For the majority of our random generations that is not the case, but if a randomly generated sound has passed this phase, our three categorizers assign their categorizations to this sound. These categorizers have a moderate degree of agree-ability as seen in ~\ref{survey}, but often the decision is not unanimous. The fourth method of categorization, "averaged-cat", is implemented by taking the sum of the softmax outputs of all three categorizers, using it to determine the category. 

These models can be combined and weighted in various ways and the confidence thresholds can be modified in order to implement "virtual ears" with different properties. A glaring issue in the current implementation is the treatment of softmax outputs as a reasonable measure of a model's confidence. Ignoring that some models may have unwarranted higher confidence in their scores, skewing attempts at finding a consensus. 


\subsection{Embedding Ears}
% http://www.justinsalamon.com/uploads/4/3/9/4/4394963/cramer_looklistenlearnmore_icassp_2019.pdf says they need almost 40 million samples 
% http://www.cs.toronto.edu/~zemel/documents/prototypical_networks_nips_2017.pdf
% 

\section{Data And Project Replication}
\label{data}

Our data is a large set of drum samples aggregated from personal libraries, free drum kits from the sample-swap project \footnote{https://sampleswap.org/} which we further processed to suit our categories, and a large set of drum sounds aggregated from royalty free sources such as musicradar \footnote{https://www.musicradar.com/}. We have made our dataset of free-drum sounds available for download. The scripts used to download and process royalty free samples will also be made available. Further information about downloading our dataset can be found on this project's github page. 

Our drum categories are claps, hats, kicks, rims/other, shakers, snares and toms. Other categories are chopped guitars, chopped pianos and n-stack-synths (random noise generated by the virtual synth with stack size of n, see ~\ref{vs}), utilized for learning percussive vs non-percussive sounds. Stack sizes refers to how many synth functions are connected together in a synthesizer.
Some notes about our dataset:


\begin{itemize}
\item Of the 6000 drum-sounds utilized in our work, the kick, snare and hat categories have the largest share at around 20\% each, while the shaker and rim (other) categories have the smallest at 5\% combined. Due to this we only focused on learning from kicks, snares, toms, claps and hats for Phase 1 of training (along with non-percussive groups of sound) ~\ref{sec:ear}.

\item For Phase 2 of training we only focused on categorizing snares, claps, kicks, hats and other (percussive sounds such as shakers, rims and unusual percussions that we couldn't categorize were grouped into this category). Non-percussive datasets were not used for this phase. 
\item In order to offset bias from data imbalance during training of our models, the categorical cross entropy loss was weighted by the group sizes. 
\item For any given model, 80\% of our data is used for training and 20\% is used for testing. 

\item We limit the size of the n-stack-synths category to 50\% of the total size of our drum dataset. This is done in order to measure whether the features extracted can address the "Open Set Recognition" problem, which will be discussed further in Section ~\ref{sec:ear}.
\end{itemize}
\end{document}