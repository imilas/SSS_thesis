% Allow relative paths in included subfiles that are compiled separately
% See https://tex.stackexchange.com/questions/153312/
\providecommand{\main}{..}

\documentclass[\main/thesis.tex]{subfiles}
\externaldocument{}



\begin{document}
\chapter{Datasets}
\label{sec:datasets}
\section{Why is Data Needed?}
This project requires a \textit{virtual synthesizer} capable of producing sounds with a variety of characteristics (as long as a fraction of these sounds can be suitable replacements for percussion). On the other-hand, the \textit{virtual ear} has a more concrete task: the separation of drum-like sounds from other synthesizer sounds. Rather than manually defining the characteristics which distinguish percussive sounds from all other types, we use supervised machine learning to train the virtual ear by example. To this end, we gathered 3 databases of percussive sounds, and use supervised machine learning methods to create our virtual ear models. Chapter~\ref{sec:ear} will cover how these datasets are transformed and used for training.


\section{Datasets}
We curated 3 different datasets of drums, as well as a dataset of randomly generated synthesizer sounds. These drums are meant to represent conventional drum sounds, which can be recordings of physical drums or designed by sound engineers using analogue synthesizers such as the~\enquote{Roland TR-808}.  The breakdown of samples in each dataset is given in Table~\ref{table:all_db}. Below is a summary of how each dataset was curated, and the steps to reproduce it, if possible. 
\subsection{FreeDB}
FreeDB is our curated dataset of free drum-kits extracted from the \enquote{SampleSwap} project\footnote{\url{https://sampleswap.org/}}. The SampleSwap database contains a variety musical and non-musical sounds. We manually selected the sub-directories from SampleSwap which contained drum sounds, and grouped drum sounds which did not belong to \enquote{Kick}, \enquote{Snare}, \enquote{Clap}, \enquote{Hat} into the group \enquote{Other}. FreeDB is copyright free and is available at: \url{https://zenodo.org/record/3994999}. 
\subsection{RadarDB}
RadarDB is a set of drum sounds aggregated from royalty free sources such as music radar\footnote{https://www.musicradar.com/}. We cannot directly share this dataset as it is not copyright free, however, the script for its automated creation can be found under the \enquote{getting\_data} directory of the project: \url{https://github.com/imilas/Synths_Stacks_Search}. Be aware that nearly 50GBs of compressed audio files will be downloaded, extracted, and filtered to create RadarDB. 

\subsection{MixedDB}
MixedDB is a large set of 2 second or shorter drum samples aggregated from personal libraries. We cannot share this dataset as it contains copyrighted material. 
\subsection{NoiseDB}
NoiseDB is our database of synthetic noise from 1, 3, and 5 stacked virtual synthesizers (synthesizer stacks will be discussed in the upcoming Section~\ref{virtual_synth_implementation}). 2000 sounds of each stack size were selected for a total of 6000 sounds.  This dataset is used as a source of~\enquote{negative examples}, i.e, sounds which we generally want to reject, unless they are very similar to drum sounds.
\subsection{FreeRadarDB}
\label{db:memDB}
A database put together by combination of radarDB, FreeDB and NoiseDB.

\section{How The Datasets Will Be Used}
 The drum and non-drum data here can be used as examples to learn the characteristics of drums and non-drums. In Section~\ref{sec:ear}, we discuss how supervised machine learning algorithms are trained to categorize drums from non-drums using these datasets.  
  
% \section{Download Instructions}
% \label{appendix:datasets}
% FreeDB, the survey data used in Section~\ref{surveys}, and randomly generated sounds can be downloaded from: \url{https://zenodo.org/record/3994999}\\\\




\begin{table}[]
% \centering
% \caption{Curated databases, including NoiseDB}
\begin{tabular}{ |c|c|c| } 
\hline
DB Name & Categories \\ \hline
FreeDB  & Kicks:533 - Snares:372 - Claps:230 - Hats:105 - Other:281            \\ \hline
RadarDB & \makecell{Kicks:1054 - Snares:842 - Claps:353 \\ Toms:349 - Hats:1561 - Rims:131 - Shakers:121} \\ \hline
MixedDB & Kicks:648 - Snares:732 - Claps:179 - Hats:105 - Toms:416 - Others:281                     \\ \hline
NoiseDB & 1 Stack:2000 - 3 Stacks:2000 - 5 Stacks:2000                         \\ \hline
 FreeRadarDB & kick:1334 - snare:1035 - clap:401 - hat:1275 - Synthetic:1000 \\ \hline
\end{tabular}
    \caption{Overview of our curated datasets.}
    \label{table:all_db}
\end{table}


% DB Name & kick & snare & clap & tom\_high & tom\_mid & tom\_low & hihat\_closed &  hihat\_open & rim \\ \hline
% MixedDB & 648 & 732 & 118 & 179 & 139 &  188 & 187 & 280 & 105 \\\hline

% \begin{table}[h]
% \centering
%  \resizebox{\linewidth}{!}{\begin{tabular}{|l|l|l|l|l|l|l|l|l|l|}
% \hline
% DB Name & kick & snare & clap & tom\_high & tom\_mid & tom\_low & hihat\_closed &  hihat\_open & rim \\ \hline
% MixedDB & 648 & 732 & 118 & 179 & 139 &  188 & 187 & 280 & 105 \\\hline
% \end{tabular}}
% \caption{Database 1: Mixed sources}
% \label{db:self}
% \end{table}

% \begin{table}[h]
% \centering
% \begin{tabular}{|l|l|l|l|l|l|l|l|l|}
% \hline
% DB Name & kick & snare & clap & tom & clap & hat & rim & shaker  \\ \hline
% RadarDB & 1054 & 842   & 353 & 349 &  353 & 1561& 131 & 121 \\ \hline
% \end{tabular}
% \caption{Database 2: Royalty free sounds sourced from \enquote{Music Radar}}
% \label{db:radar}
% \end{table}

% \begin{table}[h]
% \centering
% \begin{tabular}{|l|l|l|l|l|l|}
% \hline
%  DB Name & kick & snare & clap & hat & other \\\hline
%  FreeDB & 533 & 372 & 230 & 105 & 281 \\ \hline
% \end{tabular}
% \caption{Database 3: Free sounds sourced from the \enquote{Sample Swap} project. Simplified for our purposes. The version available for download contains more sample groups. The \enquote{other} category contains a variety of percussive sounds.}
% \label{db:sampleswap}
% \end{table}


% \begin{table}[h]
% \centering
% \begin{tabular}{|l|l|l|l|}
% \hline
%  Synth Noise Type & 1 Stack & 3 Stacks  & 5 Stacks \\ \hline
%  Number of Examples & 2000 & 2000 & 2000 \\ \hline
% \end{tabular}
% \caption{Database of random noise examples from our virtual synthesizers}
% \label{db:noise}
% \end{table}


\end{document}